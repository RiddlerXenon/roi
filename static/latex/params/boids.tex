dt
Шаг интегрирования $\Delta t$
Дискретизация времени для явной схемы Эйлера. Увеличение ускоряет процесс эволюции всей системы.

v_max
Ограничение скорости $v_{max}$
Верхняя граница для нормы скорости $\| v \|$. Явно определяет максимальную скорость движения всех  особей и косвенно ограничивает быстроту поворота без явно расчета кривизны.

a_max
Ограничение ускорения $a_{max}$
Верхняя граница для нормы результирующего ускорения после суммирования всех компонент побуждений. Определяет маневренность, подавляет резкие изменения траектории.

v_pref
Предпочтительная скорость $v_{pref}$
Целевая скорость для опорных векторов выравнивания и центрирования. Формирует типовой масштаб движения, не являясь жестким ограничением.

tauMatch
Постоянная выравнивания $\tau_{\mathrm{match}}$
Время релаксации в $\a_{\mathrm{match}}$. Уменьшение ускоряет локальное согласование скоростей.

tauCenter
Постоянная центрирования $\tau_{\mathrm{center}}$
Время релаксации в $\a_{\mathrm{center}}$. Уменьшение ускоряет быстроту переориентирования особей к локальному центру.

tauSep
Постоянная разделения $\tau_{\mathrm{sep}}$
Время релаксации в $\a_{\mathrm{sep}}$. Задает быстроту реакции на сближение.

k_sep
Интенсивность разделения $k_{\mathrm{sep}}$
Безразмерное масштабирование суммарной «социальной» силы в ближней зоне. Линейно усиливает отталкивание независимо от $\tau_{\mathrm{sep}}$

dampMode
Режим вязкости среды
Включение компоненты $a_{\mathrm{damp}} = \gamma v$ в суммарное ускорение.

gamma
Коэффициент вязкости $\gamma$
Параметр экспоненциального затухания свободного движения.

neighborMode
Схема соседства
Выбор окружения при выравнивании и центрировании: метрическое - по радиуса $r}$, топологическое - по ближайшим $k$ соседям.

r
Радиус восприятия $r$
Порог расстояния для метрического соседства. Применяется совместно с углом моделируемого поля зрения $\phi$.

kTopo
Число топологических соседей $k$
Размерность окружения при топологической схеме соседства. Не зависит от плотности агентов.

fovDeg
Угол поля восприятия $\phi$
Полный угол поля восприятия, ориентируемого относительно текущей скорости $v$. Определяет анизатропный выбор соседей.

r_sep
Радиус ближней зоны $a_{\mathrm{sep}}
Изотропная зона действия правила разделения. Не зависит от сектора $\phi$.

w.match
Вес выравнивания $w_{\mathrm{match}}$
Линейное масштабирование вклада компоненты выравнивания скоростей $a_{\mathrm{damp}}$ в суммарное ускорение.

w.center
Вес центрирования $w_{\mathrm{match}}$
Линейное масштабирование вклада компоненты центрирования $a_{\mathrm{center}}$ в суммарное ускорение.

w.sep
Вес разделения $w_{\mathrm{sep}}$
Линейное масштабирование вклада компоненты разделения $a_{\mathrm{sep}}$ в суммарное ускорение.

boidCount
Число агентов $N$
Количество особей на сцене.

tracing
След траектории
Отрисовка историй движения особей. Несет лишь визуальный характер, не влияя на динамику.

showFov
Отображения полей восприятия
Отрисовка сектора $\phi$ и окружность $r_{\mathrm{sep}}$ для отображения геометрии восприятия.
